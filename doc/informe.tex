\documentclass{article}

\begin{document}
\begin{titlepage}
  \centering
  {\bfseries\LARGE Universidad De La Habana \par}
  \vspace{1cm}
  {\Large MATCOM \par}
  \vspace{3cm}
  {\Huge Simulación \par}
  \vspace{1cm}
  {\Large Proyecto 1 }
  \vfill
  {\Large Olivia González Peña \par}
  {\Large C411 \par}

  \vfill
\end{titlepage}

\section {La cocina de Kojo (Kojo's Kitchen)}

\paragraph{}
La cocina de Kojo es uno de los puestos de comida rápida en un centro comercial. El centro comercial está abierto
entre las 10:00 am y las 9:00 pm cada día. En este lugar se sirven dos tipos de productos: sándwiches y sushi.
Para los objetivos de este proyecto se asumirá que existen solo dos tipos de consumidores: unos consumen solo
sándwiches y los otros consumen solo productos de la gama del sushi. En Kojo hay dos periodos de hora pico durante 
un día de trabajo; uno entre las 11:30 am y la 1:30 pm, y el otro entre las 5:00 pm y las 7:00 pm. El intervalo 
de tiempo entre el arribo de un consumidor y el de otro no es homogéneo pero, por conveniencia, se asumirá que es 
homogéneo. El intervalo de tiempo de los segmentos homogéneos, distribuye de forma exponencial. Actualmente dos 
empleados trabajan todo el día preparando sándwiches y sushi para los consumidores. El tiempo de preparación depende 
del producto en cuestión. Estos distribuyen de forma uniforme, en un rango de 3 a 5 minutos para la preparación de 
sándwiches y entre 5 y 8 minutos para la preparación de sushi. El administrador de Kojo está muy feliz con el 
negocio, pero ha estado recibiendo quejas de los consumidores por la demora de sus peticiones. él está interesado 
en explorar algunas opciones de distribución del personal para reducir el número de quejas. Su interés está centrado 
en comparar la situación actual con una opción alternativa donde se emplea un tercer empleado durante los periodos 
más ocupados. La medida del desempeño de estas opciones estará dada por el porciento de consumidores que espera más 
de 5 minutos por un servicio durante el curso de un día de trabajo. Se desea obtener el porciento de consumidores 
que esperan más de 5 minutos cuando solo dos empleados están trabajando y este mismo dato agregando un empleado en 
las horas pico.

\section {Solución}
\subsection {Aclaraciones}
\paragraph{}
El tiempo entre los arribos de los clientes se asume que es homogéneo y que distribuye Exp($\lambda$). Los clientes ordenan 
un único plato, el cual distribuye Ber(0.5). El tiempo de preparación de los sushis y los sándwiches distribuyen 
U(5,8) y U(3,5), respectivamente.
\paragraph{}
Se asume que el tiempo de preparación del plato ordenado por un cliente (rango de tiempo especificado en la orden) 
no afecta en la satisfacción con el servicio que recibe este. Se entiende como tiempo de espera el tiempo que 
transcurre desde que el cliente llega al puesto hasta que realiza su pedido, y este es el tiempo que se analiza 
para “juzgar” el servicio.
\paragraph{}
La jornada laboral se analiza en minutos, por lo que el horario especificado se interpreta entonces como 660 minutos. 
Se recibirá a todos los clientes que arriben en este intervalo y, una vez alcanzado el fin de la jornada, solo se 
trabajará para terminar de atender a los clientes aun en el puesto. Los horarios pico, en consecuencia, serán 
interpretados como los intervalos entre los 90 y 210 minutos, y entre los 420 y 540 minutos.

\subsection {Modelo}
\paragraph{}
El problema puede ser visto como el de atender peticiones por un conjunto de servidores en paralelo, en este caso las 
peticiones serían las ordenes generadas por los clientes y los servidores serían los trabajadores que las ejecutan. Este 
problema cuenta con la peculiaridad de que la cantidad de servidores (trabajadores) puede variar en ciertos intervalos.

La solución se resume en atender los eventos que se presentan al transcurrir el tiempo. Existen dos tipos de eventos, 
el arribo de un cliente y la salida de un cliente. En dependencia del tipo, se analiza cómo cambia este al entorno 
en que se desarrolla y, a su vez, la solución al evento está dada por la situación en que se encuentra el entorno 
llegado el momento de ser atendido.
\paragraph{}
Un primer caso sería cuando arriba un cliente y al menos uno de los trabajadores de la cocina está libre y pasaría a 
atenderlo directamente. Este es, por ejemplo, el caso inicial del problema.
\paragraph{}
Otro escenario es cuando llega un cliente y no puede ser atendido porque todos los trabajadores están ocupados con 
otros pedidos, en cuyo caso el cliente pasa a formar parte de una cola de espera. Es precisamente el tiempo que se 
encuentra un cliente en esta cola, el que nos interesa analizar para calificar la atención que se está brindando 
en el puesto.
\paragraph{}
En ambos casos, cuando un trabajador pasa a preparar el pedido de un cliente, este se encuentra ‘ocupado’ y no es 
posible que atienda el de ningún otro. En cuanto termine de preparar el pedido el cliente asociado al mismo saldrá del puesto 
dando lugar al segundo tipo de evento y el trabajador se encontrará ‘disponible’ para atender al próximo. Si en la 
cola de espera hay clientes serán atendidos estos siguiendo la filosofía FIFO, sino queda libre y estaríamos de 
nuevo en el primer caso.
\paragraph{}
La solución es la misma para los casos en los que en toda la jornada de trabajo se cuenta con dos trabajadores y para los 
casos en los que se cuenta con tres trabajadores en los horarios picos. Este último escenario se simula teniendo 
un trabajador más para los eventos que ocurren en los intervalos de tiempo que comprenden los horarios picos (El 
trabajador ‘extra’ se comporta de la misma forma para cada intervalo en el que trabaja, que los trabajadores 
permanentes en toda su jornada laboral; es decir, llegado el fin de su intervalo de trabajo no atiende más 
peticiones, pero termina las que haya empezado durante este).
\paragraph{}
El programa llega a su fin cuando han transcurrido los 660 minutos y no hay clientes en el puesto con 
pedidos pendientes.
\paragraph{}
La insatisfacción de los clientes se expresa como el promedio de clientes con este sentimiento en 1000 simulaciones 
para cada caso (contando con dos trabajadores durante toda la jornada y contando con uno extra en los horarios pico).
\paragraph{}
Tras haber realizado las pruebas para distintos valores de $\lambda$ , se obtienen los resultados siguientes:
\paragraph{}

\begin{table}[h!]
    \begin{center}
    \begin{tabular}{| c | c | c | c |}
    \hline
    \multicolumn{4}{ |c| }{Resultados con dos trabajadores} \\ \hline
    $\lambda$ & clientes atendidos & clientes insatisfechos & $ \% $ de insatisfacción \\ \hline
    0.5  & 330 & 318 & 96.3636 \\ \hline
    0.4  & 264 & 226 & 85.6060 \\ \hline
    0.3  & 198 & 65 & 32.8282 \\ \hline
    0.2 & 131 & 8 & 6.1068 \\ \hline
    0.1 & 65 & 0 & 0 \\ \hline
    \end{tabular}
    \end{center}
\end{table}

\begin{table}[h!]
  \begin{center}
  \begin{tabular}{| c | c | c | c |}
  \hline
  \multicolumn{4}{ |c| }{Resultados con tres trabajadores (horario pico)} \\ \hline
  $\lambda$ & clientes atendidos & clientes insatisfechos & $ \% $ de insatisfacción \\ \hline
  0.5  & 331 & 297 & 89.7280 \\ \hline
  0.4  & 265 & 146 & 55.0943 \\ \hline
  0.3  & 198 & 38 & 19.1919 \\ \hline
  0.2 & 132 & 5 & 3.7878 \\ \hline
  0.1 & 65 & 0 & 0 \\ \hline
  \end{tabular}
  \end{center}
\end{table}

\section {Conclusiones}
\paragraph{}
Se aprecia cómo incluso teniendo un trabajador extra en los horarios pico, si los intervalos entre los arribos de 
los clientes son muy cortos, la insatisfacción de los clientes se generaliza ($\lambda = 0.5$).

Para $\lambda = 0.4$ a pesar de que se mejora las satisfacción de los clientes en un 30\% agregando un trabajador en
horarios pico, se considera que no se obtiene un \% de satisfacción adecuado para un buen servicio.

Un caso digno de análisis ocurre cuando $\lambda = 0.3$, donde si bien no se obtiene mayoría
de clientes insatisfechos con dos trabajadores durante toda la jornada, contando con un trabajador extra en los
horarios pico, se reduce la insatisfacción en aproximadamente un 40\% con respecto al primer caso.

Sopesando el gasto que conllevaría contratar a un tercer trabajador, y teniendo en cuenta que el \%
de  insatisfacción es bastante bajo para el caso donde $\lambda = 0.2$, añadir un tercer trabajador sería
innecesario.

Analizando el otro extremo, si los 
intervalos de llegada de los clientes son mayores o iguales que los intervalos que requiere la preparación de los 
platos, incorporar a un tercer trabajador es absurdo ($\lambda = 0.1$).

\section {Ejecución}
python3 main.py $<$lambda$>$

\section {Repositorio Github}


\end{document}


